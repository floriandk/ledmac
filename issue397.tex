\documentclass{article}

\usepackage{reledmac}

\Xarrangement[A]{paragraph}

\begin{document}
\meaning\textdir
\beginnumbering
\pstart
Subiectum igitur huius scientiae est animal, secundum quod in
eo est principium motus ad somnum et uigiliam, et hoc est rationale,
quia ista scientia subalternatur
\edtext{scientiae}{\Afootnote{subiectae \emph{M}}} traditae in libro
Physicorum, et ideo subiectum huius continetur sub subiecto illius.
Somnus multipliciter describitur. Philosophus enim ponit duas
descriptiones somni; una est, quod somnus est impotentia sentiendi
propter excessum uigiliae, alia est, quod somnus est immobilitatio
sensuum \edtext{exteriorum}{\Afootnote{exteriori
        \emph{a. c. M}}}. Alias duas descriptiones ponit Commentator;
una est, quod somnus est sensus in potentia. Dormienti
\edtext{enim}{\Afootnote{\emph{om. L}}} uidetur, quod comedat et potet
et sentiat per omnes \edtext{quinque}{\Afootnote{\emph{s.  l. L,
            om. A}}} sensus, et secundum eum uigilia est sensus in
actu, et quandoque \edtext{contingit}{\Afootnote{conuenit \emph{M}}},
quod sensus in potentia exeat in actum, ut in somniis ueris
praenuntiatiuis mirabilium, et tunc sensus, qui est in potentia, est
nobilior, quam sensus, qui est in actu. Sensus autem in potentia, cum
fuerit falsus, tunc est uilis, et sensus in actu est nobilior eo. Alio
\edtext{modo}{\Afootnote{\emph{om. A}}}
\edtext{describit}{\Afootnote{descripsit \emph{A}}} Commentator somnum
sic: Somnus est \edtext{reuocatio}{\Afootnote{reuolutio \emph{A}}}
\edtext{caloris}{\Afootnote{calorum \emph{A}}} naturalis ad interius
ab \edtext{organis}{\Afootnote{orig(.) \emph{a. c. M}}} corporeis
\edtext{exterius}{\Afootnote{\emph{An} externis \emph{scribendum?}}},
et \edtext{sic}{\Afootnote{tunc \emph{AL}}} patet, quid est somnus.
\pend
\endnumbering
\end{document}
